\documentclass[11pt,]{article}
\usepackage[margin=1in]{geometry}
\newcommand*{\authorfont}{\fontfamily{phv}\selectfont}
\usepackage[]{mathpazo}
\usepackage{abstract}
\renewcommand{\abstractname}{}    % clear the title
\renewcommand{\absnamepos}{empty} % originally center
\newcommand{\blankline}{\quad\pagebreak[2]}

\providecommand{\tightlist}{%
  \setlength{\itemsep}{0pt}\setlength{\parskip}{0pt}} 
\usepackage{longtable,booktabs}

\usepackage{parskip}
\usepackage{titlesec}
\titlespacing\section{0pt}{12pt plus 4pt minus 2pt}{6pt plus 2pt minus 2pt}
\titlespacing\subsection{0pt}{12pt plus 4pt minus 2pt}{6pt plus 2pt minus 2pt}

\titleformat*{\subsubsection}{\normalsize\itshape}

\usepackage{titling}
\setlength{\droptitle}{-.25cm}

%\setlength{\parindent}{0pt}
%\setlength{\parskip}{6pt plus 2pt minus 1pt}
%\setlength{\emergencystretch}{3em}  % prevent overfull lines 

\usepackage[T1]{fontenc}
\usepackage[utf8]{inputenc}

\usepackage{fancyhdr}
\pagestyle{fancy}
\usepackage{lastpage}
\renewcommand{\headrulewidth}{0.3pt}
\renewcommand{\footrulewidth}{0.0pt} 
\lhead{}
\chead{}
\rhead{\footnotesize STAT 506: Advanced Regression Analysis -- Spring 2021}
\lfoot{}
\cfoot{\small \thepage/\pageref*{LastPage}}
\rfoot{}

\fancypagestyle{firststyle}
{
\renewcommand{\headrulewidth}{0pt}%
   \fancyhf{}
   \fancyfoot[C]{\small \thepage/\pageref*{LastPage}}
}

%\def\labelitemi{--}
%\usepackage{enumitem}
%\setitemize[0]{leftmargin=25pt}
%\setenumerate[0]{leftmargin=25pt}




\makeatletter
\@ifpackageloaded{hyperref}{}{%
\ifxetex
  \usepackage[setpagesize=false, % page size defined by xetex
              unicode=false, % unicode breaks when used with xetex
              xetex]{hyperref}
\else
  \usepackage[unicode=true]{hyperref}
\fi
}
\@ifpackageloaded{color}{
    \PassOptionsToPackage{usenames,dvipsnames}{color}
}{%
    \usepackage[usenames,dvipsnames]{color}
}
\makeatother
\hypersetup{breaklinks=true,
            bookmarks=true,
            pdfauthor={ ()},
             pdfkeywords = {},  
            pdftitle={STAT 506: Advanced Regression Analysis},
            colorlinks=true,
            citecolor=blue,
            urlcolor=blue,
            linkcolor=magenta,
            pdfborder={0 0 0}}
\urlstyle{same}  % don't use monospace font for urls


\setcounter{secnumdepth}{0}





\usepackage{setspace}

\title{STAT 506: Advanced Regression Analysis}
\author{Andrew Hoegh}
\date{Spring 2021}


\begin{document}  

		\maketitle
		
	
		\thispagestyle{firststyle}

%	\thispagestyle{empty}


	\noindent \begin{tabular*}{\textwidth}{ @{\extracolsep{\fill}} lr @{\extracolsep{\fill}}}


E-mail: \texttt{\href{mailto:andrew.hoegh@montana.edu}{\nolinkurl{andrew.hoegh@montana.edu}}} & Web: \href{http://stat506.github.io}{\tt stat506.github.io}\\
Office Hours: TBD  &  Class Hours: MWF 10:00-10:50\\
Office: Wilson Hall 2-241  & Class Room: \emph{Wilson Hall 1-144}\\
	&  \\
	\hline
	\end{tabular*}
	
\vspace{2mm}
	


\hypertarget{course-description}{%
\section{Course Description}\label{course-description}}

This course will continue wrap up linear models and generalized linear
models from STAT505, including a more detailed look at the underlying
linear algebra. In addition, the course will present advanced regression
techniques including hierarchical models.

\hypertarget{learning-outcomes}{%
\section{Learning Outcomes:}\label{learning-outcomes}}

\begin{itemize}
\tightlist
\item
  To fit hierarchical models in R and SAS and interpret the results.
\item
  To fit models which take into account common forms of correlation.
\item
  To fit models which take into account common forms of non-constant
  variance.
\item
  To make inference using models which do not assume normality of
  residuals.
\item
  To fit Bayesian models using Markov Chain Monte Carlo algorithms and
  to interpret results.
\end{itemize}

\hypertarget{additional-topics}{%
\section{Additional Topics}\label{additional-topics}}

\begin{itemize}
\tightlist
\item
  Understand the derivation of generalized least squares estimates.
\item
  To know when the Gauss-Markov theorem applies and what it provides.
\item
  To interpret results from Poisson and logistic regression models.
\item
  To understand when causal inference can be made from observational
  studies.
\end{itemize}

\hypertarget{prerequisites}{%
\section{Prerequisites}\label{prerequisites}}

\begin{itemize}
\tightlist
\item
  Required: STAT 505
\end{itemize}

\hypertarget{textbooks}{%
\section{Textbooks}\label{textbooks}}

\begin{itemize}
\tightlist
\item
  \emph{Regression and Other Stories}, by Andrew Gelman, Jennifer Hill,
  and Aki Vehtari
\item
  \emph{Data Analysis Using Regression and Multilevel/Hierarchical Models},
  by Andrew Gelman and Jennifer Hill
\end{itemize}

\hypertarget{additional-resources}{%
\section{Additional Resources}\label{additional-resources}}

Analysis, data visualization, and version control procedures will be
implemented with:

\begin{itemize}
\tightlist
\item
  R / R Studio
\item
  Git / Github
\end{itemize}

For additional resources see:

\begin{itemize}
\tightlist
\item
  R for Data Science, \url{https://r4ds.had.co.nz}
\item
  Happy Git and GitHub for the useR, \url{https://happygitwithr.com}
\end{itemize}

\hypertarget{course-policies}{%
\section{Course Policies}\label{course-policies}}

\hypertarget{grading-policy} of your grade will be determined by weekly homework
  assignments. Students are allowed and encouraged to work with
  classmates on homework assignments, but each student is required to
  complete their own homework.
\item
  \textbf{30\%} of your grade will be determined by a final exam.
\item
  \textbf{40\%} of your grade will be determined by a series of
  projects. The projects will be due roughly every 3 - 4 weeks during
  the course of the semester and may require oral presentations.
\end{itemize}

\hypertarget{collaboration}{%
\subsubsection{Collaboration}\label{collaboration}}

University policy states that, unless otherwise specified, students may
not collaborate on graded material. Any exceptions to this policy will
be stated explicitly for individual assignments. If you have any
questions about the limits of collaboration, you are expected to ask for
clarification.

In this class students are encouraged to collaborate on homework
assignments, but exams and projects should be completed without
collaboration.

\hypertarget{academic-misconduct}{%
\subsubsection{Academic Misconduct}\label{academic-misconduct}}

Section 420 of the Student Conduct Code describes academic misconduct as
including but not limited to plagiarism, cheating, multiple submissions,
or facilitating others' misconduct. Possible sanctions for academic
misconduct range from an oral reprimand to expulsion from the
university.

\hypertarget{disabilities-policy}{%
\subsubsection{Disabilities Policy}\label{disabilities-policy}}

Federal law mandates the provision of services at the university-level
to qualified students with disabilities. If you have a documented
disability for which you are or may be requesting an accommodation(s),
you are encouraged to contact the Office of Disability Services as soon
as possible.

\hypertarget{masks}{%
\subsubsection{Masks}\label{masks}}

WEARING MASKS IN CLASSROOMS IS REQUIRED Face coverings that cover the
mouth and nose are required in all indoor spaces and all enclosed or
partially enclosed outdoor spaces. MSU requires all students to wear
face masks or cloth face coverings in classrooms, laboratories and other
similar spaces where in-person instruction occurs. MSU requires the
wearing of masks in physical classrooms to help mitigate the
transmission of SARS-CoV-2, which causes COVID-19. The MSU community
views the adoption of these practices as a mark of good citizenship and
respectful care of fellow classmates, faculty, and staff.

The complete details about MSU's mask requirement can be found at
\url{https://www.montana.edu/health/coronavirus/index.html}.

These requirements from the Office of the Commissioner of Higher
Education are detailed in the MUS Healthy Fall 2020 Guidelines, Appendix
B.

For more information:
\url{https://www.montana.edu/health/coronavirus/prevention/index.html}

Compliance with the face-covering protocol is expected. If a you do not
comply with a classroom rule, you may be requested to leave class.
Section 460.00 of the MSU Code of Student Conduct covers ``disruptive
student behavior.''

\hypertarget{health-related-class-absences}{%
\subsubsection{Health-Related Class
Absences}\label{health-related-class-absences}}

Please evaluate your own health status regularly and refrain from
attending class and other on-campus events if you are ill.~MSU students
who miss class due to illness will be given opportunities to access
course materials online. You are encouraged to seek appropriate medical
attention for treatment of illness. In the event of contagious illness,
please do not come to class or to campus to turn in work. Instead notify
me by email about your absence as soon as practical, so that
accommodations can be made. Please note that documentation (a Doctor's
note) for medical excuses is not required. MSU University Health
Partners - as part their commitment to maintain patient confidentiality,
to encourage more appropriate use of healthcare resources, and to
support meaningful dialogue between instructors and students - does not
provide such documentation.

\hypertarget{course-communication}{%
\subsubsection{Course Communication}\label{course-communication}}

In the event that one or more students and/or the instructor are
required to quarantine or if the university moves courses online, the
course may need to continue in a virtual format. Communication about how
the course will proceed will be available through D2l.

\hypertarget{virtual-attendance}{%
\subsubsection{Virtual Attendance}\label{virtual-attendance}}

Due to the ongoing pandemic and issues stemming from this, a synchronous
virtual attendance option will be permitted for this course. The
Microsoft Teams platform will be used for this virtual option. When
attending virtually, if at all possible, please plan to have your video
camera turned on.

\hypertarget{approximate-course-outline}{%
\section{Approximate Course Outline}\label{approximate-course-outline}}

\begin{enumerate}
\def\labelenumi{\arabic{enumi}.}
\tightlist
\item
  LM Recap + GLM Review
\item
  Linear Algebra Section
\item
  Design and Sample Size Decisions
\item
  Advanced Regression Overview
\item
  Hierarchical Models
\item
  Causal Inference
\end{enumerate}




\end{document}

\makeatletter
\def\@maketitle{%
  \newpage
%  \null
%  \vskip 2em%
%  \begin{center}%
  \let \footnote \thanks
    {\fontsize{18}{20}\selectfont\raggedright  \setlength{\parindent}{0pt} \@title \par}%
}
%\fi
\makeatother